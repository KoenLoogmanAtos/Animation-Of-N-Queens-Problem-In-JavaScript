% !TeX root = ./documentation.tex

\chapter{Implementation}
\label{ch:implementation}
As mentioned before in the chapter \ref{ch:tecBasics} we will be using Node.js with Parcel as a module bundler. This way we can use Node.js modules and build an HTML page that does not require a Node.js server running in the background, making it portable and executable without any prerequisites except for a browser. The implementation in JavaScript will be an object oriented approach using the class syntax introduced in ECMAScript 2015, which will make it more pleasing for the eye and less confusing \cite{MozillaDevelopers2019}.

Since the Davis-Putnam algorithm can have tasks with a high computational effort, we decided to use separate threads to compute and animate. So we can display all the changes to the problem as it is being satisfied and prevent the GUI from freezing on long lasting computations. As previously explained in chapter \ref{sec:tecWorker} this is possible with a so-called worker in JavaScript. It is important to mention here, that because the worker and main script communicate with each other via messages, we need to make sure that while computing no unwanted messages are send. Such messages could be multiple computing request being send almost simultaneously to the worker, because the worker will buffer send messages and respond to them after finishing all previous tasks. This could lead to problems if for example during a long lasting computation task additional computations are requested, then no other requests could be handled before all those computation tasks have been finished nor could those be interrupted. Our approach here will be to tell the main script when a computational task started and ended. This way certain UI elements can be locked while the worker computes and unlocked when the worker finishes the given task.

Next we need to take a look at the mathematical part of the script. Because JavaScript uses references when comparing objects, we will use Immutable.js as discussed in chapter \ref{sec:tecImmutable} to counter this problem when working with sets. Since we will have a set of clauses for the algorithm to satisfy with those clauses being sets of literals, we would need to implement a deep compare ourselves if we were to use native JavaScript. Now that we know that Immutable.js will solve the set comparison problem, the literals still need to be implemented somehow. We decided to use simple strings for the literals with the restriction of the first character not being allowed to be an ``!'', as we will use that character to indicate that the literal is negated. The choice to use strings is based on the same reason for using Immutable.js for the sets. Immutable.js has immutable structures that are compared by value, but for other objects Immutable.js compares them just like native JavaScript by their object references. This means if we were to use custom classes for literals we would need to implement a deep compare anyway, but strings on the other hand are compared by length and character order by default. Resulting in less effort when implementing the mathematical part of the algorithm \cite{MozillaDevelopers2019a}, \cite{MozillaDevelopers2019b}, \cite{ImmutableDevelopers2019}.
% Equal operator https://developer.mozilla.org/en-US/docs/Web/JavaScript/Reference/Operators/Comparison_Operators#Equality_operators
% Object is https://developer.mozilla.org/en-US/docs/Web/JavaScript/Reference/Global_Objects/Object/is
% Immutable is https://immutable-js.github.io/immutable-js/docs/#/is

So for example the formula $(\; p\; \lor\; \neg q\; )\; \land\; (\; s\; \lor\; r\; )$ would be implemented as the following Listing \ref{code:setOfClauses} with \texttt{Set} being the Set class of Immutable.js and not the native JavaScript one.

\begin{listing}[H]
\begin{minted}{javascript}
// (p or not q) and (s or r) as a set of clauses in JavaScript
let set_of_clauses = new Set(new Set('p', '!q'), new Set('s', 'r'));
\end{minted}
    \caption{Example for a set of clauses in JavaScript}
    \label{code:setOfClauses}
\end{listing}

Now that it is clear how a set of clauses will be implemented in JavaScript and the decision to use two threads has been made, we can start getting into the classes and their tasks.

\section{Util Object}
\label{sec:impUtil}
Because some functions like negating a literal will be used in multiple parts of the script, we will use an object to host these functions. This object will be called the Util Object. By doing so it can be imported when those functions are needed and prevent redundant code, making it easy to maintain or change.

It contains the function to negate a literal and functions for formatting sets and literals to HTML strings. It can be expanded further in future, but for now these will do.
Because we are using strings for literals we can negate them by adding a ``!'' and applying a simple RegEx operation afterwards as shown in the following Listing \ref{code:negateLiteral}.

\begin{listing}[H]
\begin{minted}{javascript}
Util.negateLiteral = (literal) => return ('!' + literal).replace(/^!!/, '');
Util.negateLiteral('p');  // result: !p
Util.negateLiteral('!p'); // result: p
\end{minted}
    \caption{Example for negating a literal in JavaScript}
    \label{code:negateLiteral}
\end{listing}

Formatting sets and literals to HTML, using the span element and adding classes to it, will allow us to color code them in the GUI. This will make it easier for the user to interpret the visuals of the animation.

\section{Implementing the Algorithm of Davis-Putnam}
\label{sec:impDavisPutnam}
The original \textit{Davis-Putnam algorithm} is recursive and so are most of its implementations. If you were to calculate everything till you find a possible solution or the response that the given problem can not be satisfied, then a recursive implementation is not an issue. For our purposes, which is the visualization of the algorithm as it solves the given problem step by step, we prefer to use an iterative implementation. If we use a recursive implementation we would need to calculate everything upfront, which is not the case for a iterative implementation. These can be paused at any point and because of that we can calculate the step when we need it to be calculated. This means only the creation of the problem that will be satisfied by the algorithm will affect the initial loading time.

One of the challenges here is the double recursive call of the algorithm and splitting it into single unit cuts per iteration for the micro steps. The basic idea will be a class that holds the current state of the problem with a step function that does either a macro or micro step further satisfying the problem till it is satisfied or not satisfiable. One macro step will include all micro steps till the algorithm guesses the next literal, with micro steps being single unit cuts. To be able to do this the class also needs to know what has been done previously to continue from that point. With that in mind the following points will be able to describe the state of the algorithm, so that it can continue from where it stopped.

\begin{enumerate}
    \item The set of clauses describing the current state of the problem
    \item The literals it has guessed previously
    \item The unit clauses it has already used for unit propagation
    \item The unit clause or literal that is currently in use for unit cuts
\end{enumerate}

The last point is important for the micro steps, because we need to do all possible unit cuts before using the next unit clause. Because it is possible to represent the state of the algorithm, we can also use this to simulate the double recursive call. By pushing the current state with the negation of the guessed literal to the stack and adding the guessed literal to the current state for further satisfaction, we can pop a state from the stack to continue from that point if the current state is not satisfiable.

Because we will use an iterative implementation, we need to think about everything that needs to be done in one step of the iteration. Since we will have macro and micro steps, we will use two loops one for the macro steps and one for the micro steps, so that one macro step will execute all micro steps that it includes. Lets say we have a function step that receives a number of steps $n$ so that it will do $n$ micro or macro steps on the state. Then we need to reduce the step count when a guessed literal $L$ has been added to the state, in case of macro steps, or a unit cut has been calculated, for micro steps. The function step would look like the pseudo code of Listing \ref{code:stepDavisPutnam}. Doing all micro steps at the start of a macro step results in checking if it is satisfied. If it is not satisfiable, it is followed by a backtrack and the guessing of a new literal.

\begin{listing}[h!]
    \textit{/* do step(s) on a state that contains a set of clauses $S$ */}\\
    \textbf{function} step( number of steps $n$ ) \textbf{return}\\
        \hspace*{0.5cm} \textbf{repeat} \textit{/* macro step loop */}\\
            \hspace*{1.0cm} \textbf{repeat} \textit{/* micro step loop for subsume and unit cuts */}\\
                \hspace*{1.5cm} \textit{/* select a new literal and subsume till unit cuts can be done with said literal */}\\
                \hspace*{1.5cm} \textbf{repeat}\\
                    \hspace*{2cm} choose a unit clause $U$ that has not been used yet\\
                    \hspace*{2cm} add $U$ to used unit clauses of the state\\
                    \hspace*{2cm} choose and remember literal $L$ from $U$\\
                    \hspace*{2cm} \textit{/* unit subsumption */}\\
                    \hspace*{2cm} delete from $S$ every clause containing $L$\\
                \hspace*{1.5cm} \textbf{until} all unit clauses have been used or clauses in $S$ contain $\bar{L}$\\
                \hspace*{1.5cm} \textit{/* check if cuts can be done before trying to do a unit cut */}\\
                \hspace*{1.5cm} \textbf{if} clauses in $S$ contain $\bar{L}$ \textbf{then}\\
                    \hspace*{1.5cm} \textit{/* single unit resolution */}\\
                    \hspace*{2cm} choose one of the smallest clauses $C$ that contains $\bar{L}$\\
                    \hspace*{2cm} delete $\bar{L}$ from $C$\\
                    \hspace*{2cm} \textbf{if} it is a micro step \textbf{then}\\
                        \hspace*{2.5cm} $n \texttt{-=} 1$\\
                    \hspace*{2cm} \textbf{fi}\\
                \hspace*{1.5cm} \textbf{fi}\\
            \hspace*{1.0cm} \textbf{until} all unit clauses have been used or $n = 0$\\
            \hspace*{1.0cm} \textit{/* check if satisfied */}\\
            \hspace*{1.0cm} \textbf{if} $S$ contains only unit clauses \textbf{then}\\
                \hspace*{1.5cm} \textit{/* is satisfied */}\\
                \hspace*{1.5cm} \textbf{continue}\\
            \hspace*{1.0cm} \textbf{else if} $n = 0$ \textbf{then}\\
                \hspace*{1.5cm} \textbf{continue}\\
            \hspace*{1.0cm} \textbf{fi}\\
            \hspace*{1.0cm} $n \texttt{-=} 1$\\
            \hspace*{1.0cm} \textit{/* check if current state is not satisfiable */}\\
            \hspace*{1.0cm} \textbf{if} a clause becomes $\{\}$ in $S$ \textbf{then}\\
                \hspace*{1.5cm} \textbf{if} stack is empty \textbf{then}\\
                    \hspace*{2cm} \textit{/* is not satisfiable */}\\
                    \hspace*{2cm} set $S = \{\;\{\}\;\}$\\
                    \hspace*{2cm} \textbf{continue}\\
                \hspace*{1.5cm} \textbf{fi}\\
                \hspace*{1.5cm} pop state from the stack\\
                \hspace*{1.5cm} \textbf{continue}\\
            \hspace*{1.0cm} \textbf{fi}\\
            \hspace*{1.0cm} choose and remember a literal $L$ occurring in $S$\\
            \hspace*{1.0cm} push state with literal $\bar{L}$ to the stack\\
            \hspace*{1.0cm} add literal $L$ to the current state\\
        \hspace*{0.5cm} \textbf{until} no further steps can be done or $n = 0$\\
        \hspace*{0.5cm} \textbf{return}\\
    \textbf{end function}\\
    \caption{Iterative step for \textit{Davis-Putnam algorithm}}
    \label{code:stepDavisPutnam}
\end{listing}

\subsection{Davis-Putnam Class}
\label{sub:impDavisPutnam}
As previously explained we need a stack of states to provide an iterative implementation of the \textit{Davis-Putnam algorithm}. For this purpose we create a Davis-Putnam class. Its primary features are that it contains a state, a stack of states and a function to execute a macro or micro step on the state. Because the algorithm is supposed to be animated we have to pass the operations on the set of clauses and the unit clauses of the set of clauses to represent the state. This can be done with event listeners, that are called on the important operations of the algorithm. Since there are a number of events we decided to use a class that contains functions to be called when those events happen. These events are as follows:

\begin{enumerate}
    \item set of clauses satisfied
    \item set of clauses not satisfiable
    \item backtrack
    \item choose a literal
    \item subsume
    \item unit cut
\end{enumerate}

Each of those events calls will receive the important information needed to describe the operation and state of the set of clauses. They all receive the literals of the unit clauses, so we can represent the state at those points. The first three do not need any additional information, but the later do. When a literal is chosen said literal will also be handed to the function for the event. The function for subsume will receive all removed clauses and the function for unit cut receives the used literal and the clause before and after the operation. This way all operations during one step can affect the animation.

Since the animation is for educational purposes, it is also important to be able to replicate the results. Therefor we will be using Seedrandom.js as mentioned in chapter \ref{sec:tecSeed} at this point. When choosing a random element in this class we will use our own random number generator that we initialize with a seed. This way each seed has its unique repeatable calculation. It is important to mention here that this is also only possible because of the way the Set class from Immutable.js works. ``When iterating a Set, the entries will be (value, value) pairs. Iteration order of a Set is undefined, however is stable. Multiple iterations of the same Set will iterate in the same order \cite{ImmutableDevelopers2019a}.'' % Immutable.js Set https://immutable-js.github.io/immutable-js/docs/#/Set

\subsection{Davis-Putnam Consumer Class}
\label{sub:impDavisPutnamConsumer}
This class is for handling the events of the Davis-Putnam class. It contains the following functions:

\begin{enumerate}
    \item onSatisfied
    \item onNotSatisfiable
    \item onBacktrack
    \item onChoose
    \item onSubsume
    \item onUnitCut
\end{enumerate}

Those functions are basically placeholders, since the class is supposed to be extended by a subclass. The subclass should alter these functions to their needs. Other than that this class has no further purpose.

\subsection{Davis-Putnam Worker Class}
\label{sub:impDavisPutnamWorker}
As said before we will have a worker for the calculations. This class initializes a worker for the main script to do all the calculations. It extends the Davis-Putnam Consumer from the previous chapter \ref{sub:impDavisPutnamConsumer}. It will create its own Davis-Putnam instance and add itself as a consumer, so that all events from said instance will call the functions of this class. When these functions are called the worker will send messages to the main script with all the information. In addition to those messages it will also tell the main script when it starts or stops calculating, because of the potential problem named previously in chapter \ref{ch:implementation}. These will be all outgoing messages. As for the incoming messages from the main script, we will accept requests for a new set of clauses, a seed for the Davis-Putnam class, switching between micro and macro steps, and to calculate the next step.

Because the worker and main script communicate via messages only, there is only one listener for all those events. To be able to distinguish between the different requests we will define a default message structure as shown in the following Listing \ref{code:workerMessage}.

\begin{listing}[h!]
\begin{minted}{javascript}
// message example
let message = {
    'cmd': command_name,
    'options': {
        'option_1': value_1,
        'option_2': value_2
    }
}
\end{minted}
    \caption{Worker message example}
    \label{code:workerMessage}
\end{listing}

The ``cmd'' value will indicate what kind of request it is and the ``options'' object will contain all the information needed to fulfill that request. This way the listener can be implemented with a simple switch-case to handle the requests as shown in the following Listing \ref{code:workerListener}.

\begin{listing}[h!]
\begin{minted}{javascript}
// listen to events
self.addEventListener('message', event => {
    let cmd = event.data.cmd;
    let options = event.data.options;

    // handle commands
    switch(cmd) {
        case 'seed':
            // handle seed change
            break;
        case 'clauses':
            // handle clauses change
            break;
        case 'micro':
            // handle micro/macro change
            break;
        case 'step':
            // handle step
            break;
        default:
            // undefined command
    }
});
\end{minted}
    \caption{Event listener of the Davis-Putnam Worker class}
    \label{code:workerListener}
\end{listing}

With the components of the calculation thread finished, we are only missing those of the animation thread.

\section{Implementing the N-Qeens Problem}
\label{sec:impQueens}
To be able to solve a problem with the \textit{Davis-Putnam algorithm}, a propositional formula needs to be defined in JavaScript. Therefore we will create a function to return for a given number $n$ a set of clauses that express the \textit{N-Queens Problem} for said $n$. The queens will be represented by a literal of the form ``row,col'' and the negated literal will mean that no queen can be placed there. Lets say for example we have $\texttt{row} := 3 \land \texttt{col} := 4$ then the literal for the queen would be as follows ``3,4'' and the negation would be ``!3,4''. This makes it easy to transform the string back into coordinates for the chessboard.

\subsection{Queens Clauses Function}
\label{sub:impQueensClauses}
This function returns a set of clauses for the \textit{N-Queens Problem}, so that for a given $n$ the problem will be to place $n$ queens on a $n \times n$ board. To create this set of clauses the following functions are used to help:

\paragraph{atMostOne}
Returns a set of clauses that allows only one of the given literals to be true, for a given set of literals.

\paragraph{atMostOneInRow}
Returns a set of clauses where only one of the row can be true, for a given $n$ and $row$.

\paragraph{oneInColumn}
Returns a set of clauses where at least one in the column is true, for a given $n$ and $column$.

\paragraph{atMostOneInUpperDiagonal}
Returns a set of clauses where only one in the upper diagonal can be true, for a given $n$ and helper variable $k$.

\paragraph{atMostOneInLowerDiagonal}
Returns a set of clauses where only one in the lower diagonal can be true, for a given $n$ and helper variable $k$.

\subsection{Chessboard Class}
\label{sub:impChessboard}
To visualize the \textit{N-Queens Problem}, we will use Two.js as mentioned before in chapter \ref{sec:tecTwo}. The Chessboard class will receive a number $n$ and create a visual representation of a $n \times n$ chessboard. Since the state of the set of clauses will be represented by the literals of the unit clauses, we will need two different indicators per tile of the board. We decided to represent the queen by a green dot and a negative literal by a red cross.

The chessboard will have four layers the first being the board itself, the second for the $n \times n$ tiles of the board, the third for the queens and the last for the crosses. This way we can initialize the board with all needed Two.js objects and add or remove them from their specific layers. To only add or remove changes between two states, the chessboard will remember the state to be able to compare it with the new state. This will result in a smoother animation and less visual changes.

As for scaling of the chessboard, we will scale the Two.js scene to fit the parent HTML container with a listener that gets called when the window resizes.

\section{Implementing User Interaction}
\label{sec:impUI}
Now that we have the \textit{Davis-Putnam algorithm} and \textit{N-Queens Problem} set up in JavaScript, we need to enable the user to interact with it. In chapter \ref{ch:conception} we talked about the design. Said design will be used as a reference for the GUI of the animation. Therefor we need an HTML, CSS and JavaScript file for the webpage.

As for the interaction of the User with the animation, we will enable the user to do the following:

\begin{enumerate}
    \item change the seed for the Davis-Putnam class,
    \item change the number $n$ for the \textit{N-Queens Problem} affecting also the visualization,
    \item toggle between micro and macro steps,
    \item request the next step to be animated,
    \item request to automatically animate all steps until requested to stop or no further steps can be done,
    \item reset the whole animation to run it again from the start,
    \item display previous calculated states of the shown steps.
\end{enumerate}

For all points except the last one we got corresponding elements for interaction incorporated into the design. To enable the user to display previous states, we decided to make the elements of the calculation history clickable.

\subsection{HTML Document}
\label{sub:impHTML}
The HTML file contains all important elements needed for the GUI with their unique ids to access them via JavaScript. No additional GUI elements will be added with JavaScript except for populating the calculation log. For changing GUI elements, we will be using the ``checkbox hack''. This way we do not need to change the style of the GUI with JavaScript and additional listeners, focusing the JavaScript part only on the interaction with the worker and visual representation by the chessboard \cite{Coyier2012}.
% checkbox hack https://css-tricks.com/the-checkbox-hack/

\subsection{CSS Document}
\label{sub:impCSS}
For the layout we decided to use a grid based design, to be able to make it somewhat responsive. We added only one break-point for the smallest size we want the GUI to have, so that the chessboard has an acceptable minimal size. For larger screens it will scale up accordingly.

\subsection{Frame Class}
\label{sub:impFrame}
To initialize the animation and all its needed components, we created a Frame class. This class contains all the functions for the user interaction, the worker for the calculations and the chessboard object for visualization. It creates all the listeners for the UI elements and handles the communication with the worker. To synchronize the animation with the worker the user interface elements that should not be triggered will be disabled during the calculation process of the worker and enabled at the end of said calculation. This way we prevent unwanted behavior of the worker. Whenever the worker sends a message containing an important step of the calculation a new entry will be created for the calculation history. New entries will be added before the previous entries, so that the top entry is the latest and the bottom one the first calculation step.

Each of the named possible user interactions receives their own listener, that handles the requested task. On top of those we need one more for the messages of the worker, because we will have a two way communication. This listener is very similar to the one for the worker as previously shown in Listing \ref{code:workerListener}, only that it will handle all outgoing messages of the worker.

\section{Result}
\label{sub:impResult}
The result of the implementation can be viewed and interacted with at \href{https://koenloogman.github.io/Animation-Of-N-Queens-Problem-In-JavaScript/}{GitHub Pages of koenloogman (https://koenloogman.github.io/Animation-Of-N-Queens-Problem-In-JavaScript/)}. The animation is optimized for the latest version of Google Chrome or Firefox and maybe works on other browsers as well, but we recommend said two browsers for the best experience. If interested in the full source code, it can be seen at \href{https://github.com/koenloogman/Animation-Of-N-Queens-Problem-In-JavaScript}{GitHub of koenloogman (https://github.com/koenloogman/Animation-Of-N-Queens-Problem-In-JavaScript)}.