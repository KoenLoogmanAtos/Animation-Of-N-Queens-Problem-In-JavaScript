% !TeX root = ./documentation.tex

\chapter{Implementation}
\label{ch:implementation}
As mentioned before in the chapter \ref{ch:tecBasics} we will be using Node.js with Parcel as a module bundler. This way we can use Node.js modules and build an HTML page that does not require a Node.js server running in the background, making it portable and executable without any prerequisites except for a browser. The implementation in JavaScript will be object orientated approach using the with ECMAScript 2015 introduced class syntax, which will make it more pleasing for the eye and less confusing.
% Class Syntax https://developer.mozilla.org/en-US/docs/Web/JavaScript/Reference/Classes

\section{Util Class}
\label{sec:impUtil}

\section{Implementing the Algorithm of Davis \& Putnam}
\label{sec:impDavisPutnam}
The original algorithm of Davis \& Putnam is recursive and so are most of its implementations. If you were to calculate everything at one till you find a possible solution or the response that the given problem can not be satisfied, then a recursive implementation is not an issue. For our purposes, which is the visualization of the algorithm as it solves the given problem step by step, we prefer to use a iterative implementation. Because an iterative implementation can be paused at any point depending on how its implemented.

One of the challenges here is the double recursive call of the algorithm

\subsection{Davis Putnam Class}

\subsection{Davis Putnam Consumer Class}

\subsection{Davis Putnam Worker Class}

\section{Implementing N-Qeens Problem}
\label{sec:impQueens}

\subsection{Queens Clauses Function}

\subsection{Chessboard Class}

\section{Implementing User Interaction}
\label{sec:impUI}

\subsection{HTML Document}

\subsection{CSS Document}

\subsection{Frame Class}