% !TeX root = ../documentation.tex

% Davis Putnam in own word

\chapter{Scientific Basics}
\label{ch:sciBasics}
% cSpell:disable
% Das Ziel dieser Arbeit besteht in der Visualisierung des \textit{Davis \& Putman algorithm}us, der das sogenannte N damen Problem lösen soll. Dafür muss zuerst ein allgemeines Verständnis zu diesem Algorithmus und dem mathematischen Problem geschaffen werden. In diesem Kapitel wird aus diesem Grund dieses fundamentale Wissen zusammengefasst, um eine Basis für die weiterführende Entwicklung zu schaffen. Dafür spielt unter anderem die Deklarierung des mathematischen Problems eine Rolle, damit dieses vom \textit{Davis \& Putman algorithm}us gelöst werden kann.
% cSpell:enable
The purpose of this work is to visualize the \textit{Davis \& Putman algorithm} solving the so-called \textit{N-Queens Problem}. Therefore a general understanding of this algorithm and the mathematical problem has to be created.

This chapter summarizes the fundamental knowledge that is needed in order to create a basis for further development. Among other things, the definition of the mathematical problem plays a role here, so that a possible solution can be found by the \textit{Davis \& Putman algorithm}.

\section{Basics of Propositional Logic}
\label{sec:sciProLogic}
Before the algorithm can be examined in more detail, a few mathematical terms must first be briefly explained.

\paragraph{Propositional logic interpretation}
The function $\mathcal{I}:\mathcal{P} \rightarrow \mathbb{B}$ is called propositional logic interpretation, which assigns a true value $\mathcal{I}(p) \in \mathbb{B}$ to each proposition-logical variable $p\in \mathcal{P}$

\paragraph{Satisfiable}
There is a set of propositional logic formulas. If there is an interpretation $i$ so that all formulas are satisfied and the formula $\mathcal{I}(f) = \texttt{True}$ \quad for all propositional logic formulas in the set is given, so this set is satisfiable. If this is not the case, then is this set unsatisfiable.

\paragraph{Tautology}
It is the case, when a propositional logic formula for every interpretation $\mathcal{I}$ always true is: $\mathcal{I}(f) = \texttt{True}$. In this case it is written as $\models f$.

\paragraph{literal}
A propositional logic formula $f$ is called literal, if one of the following cases is true and $p$ is a propositional-logical variable.

\begin{enumerate}
  \item $f = \top$ or $f = \bot$
  \item $f = p$ If this is the case, it is called as a positive literal.
  \item $f = \neg p$ So it is called as a negative literal.
\end{enumerate}

\paragraph{clause}
A propositional logic formula $C$ is called clause if it has the following structure.
\\[0.2cm]
\hspace*{1.3cm} $C = l_1 \vee \cdots \vee l_r$ \\[0.2cm] At this formula, $l$ is a literal for every value $i \in \{1,\; ...,\; r\}$.

\section{The Algorithm of Davis \& Putnam}
\label{sec:sciDavisPutnam}
% cSpell:disable
% Der \textit{Davis \& Putman algorithm}us ist ein Verfahren zur Berechnung einer Lösung von aussagelogischen Klauselmengen. Bei sehr kleinen Klauselmengen kann dies leicht bestimmt werden, wie in den zwei folgenden Beispielen zu sehen ist.
% cSpell:enable
The \textit{Davis \& Putman algorithm} is a method for calculating a solution a set of propositional clauses. With very small clause sets this can be easily determined, as can be seen in the following two examples.
\\[0.2cm]
\hspace*{1.3cm} $K_1 = \bigl\{\; \{r\},\; \{\neg s\},\; \{t\},\; \{\neg u\}, \; \{\neg v\} \;\bigr\}$ 
\\[0.2cm]
% cSpell:disable
%kann auch als aussagelogische Formel geschrieben werden.
% cSpell:enable
$K_1$ can also be written as a propositional logic formula.
\\[0.2cm]
\hspace*{1.3cm} $r \land \neg s \land t \land \neg u \land \neg v$.
\\[0.2cm]
% cSpell:disable
% Es ist erkennbar, dass diese Formel lösbar ist, indem $r$ und $t$ \enquote{wahr} und $s$, $u$ und $v$ den Wert \enquote{falsch} haben. Als Gegenbeispiel dazu ist $K_2$ zu betrachten.
% cSpell:enable
It is recognizable that this formula is solvable by $r$ and $t$ \enquote{true} and $s$, $u$ and $v$ having the value \enquote{false}.
As a counter example $K_2$ is to be considered.
\\[0.2cm]
\hspace*{1.3cm} $K_2 = \bigl\{\; \{r\},\; \{\},\; \{t\} \;\bigr\}$
\\[0.2cm]
$K_2$ can also be written as a propositional logic formula.
\\[0.2cm]
\hspace*{1.3cm}
$r \land \bot \land t$.
\\[0.2cm]
% cSpell:disable
% Eine leere Klammer bedeutet in der Aussagelogik ein Falsum, wodurch $K_2$ unerfüllbar ist. Bei sehr großen Klauselmengen ist es meist nicht mehr auf dem ersten Blick zu sehen, so dass Algorithmen wie dieser eingesetzt werden. Doch um nun einen Schritt tiefer setzen zu können, müssen zwei Definitionen zuvor eingeführt werden.
% cSpell:enable
The empty set in the set of clauses $K_2$ is interpreted as a $\bot$ in the propositional logic, making it impossible to satisfy $K_2$. It is possible to satisfy smaller sets of less complex clauses by first glance, but it gets more difficult for larger and more complex sets of clauses. For these algorithms like the \textit{Davis \& Putman algorithm} are used. But before we get to the details, two definitions have to be introduced first. \cite{Zhang2000}

\paragraph{Unit clause}
% cSpell:disable
% Eine Klausel $C$ ist eine Unit-Klausel, wenn diese nur aus einem Literal, also aus einer Aussagevariablen besteht.
% cSpell:enable
A clause $C$ is a unit clause if it consists of only one literal, i.e. a variable or a negated variable.

\paragraph{trivial clause sets}
% cSpell:disable
% Eine triviale Klauselmenge kann nur vorkommen, wenn einer der beiden Fälle eintritt.
% cSpell:enable
A trivial set of clauses can only occur if one of the following two cases occurs.

\begin{enumerate}
  % cSpell:disable
  % $K$ enthält die leere Klausel und ist somit unerfüllbar
  % cSpell:enable
  \item A set of clauses $K$ contains the empty set and is therefore unsatisfiable.
  % cSpell:disable
  % Die Unit-Klauseln beinhalten immer verschiedene Aussagevariablen, so dass entweder nur die Klausel $\{p\}$ oder $\{\neg p\}$ vorkommen kann. Ist dies der Fall, kann eine Lösung für die Klauselmenge bestimmt werden.
  % cSpell:enable
  \item The unit clauses always contain different statement variables, so that only the clause $\{p\}$ or $\{\neg p\}$ can occur. If this is the case, a solution for the clause set can be determined.
\end{enumerate}

% cSpell:disable
% Damit einer dieser beiden Fälle nun eintreten kann, müssen die Klauselmengen mit der Hilfe folgender drei Möglichkeiten so vereinfacht werden, dass diese nur aus Unit-Klauseln bestehen.
% cSpell:enable
In order for one of these two cases to occur, the clause sets must be simplified with the help of the following three options so that they consist only of unit clauses.

\begin{enumerate}
  \item Cut Rule 
  \item Subsumption
  \item Case Differentiation
\end{enumerate}

\subsection{Simplification with the Cut Rule}
\label{sub:sciDavisPutnamCutRule}
To simplify a set of clauses $K$ with $C_1 \cup \{\; p\; \}\, C_2 \cup \{\; \neg p\; \} \in K$ the cut rule can be used. Lets say we have two clauses $C_1$, $C_2$ and a variable $p$, then a typical application of the cut rule is.
\\[0.2cm]
\hspace*{1.3cm} $\dfrac{C_1 \cup \{\; p\; \} \;\;\;\;\;\; C_2 \cup \{\; \neg p\; \}}{C_1 \cup C_2}$
\\[0.2cm]
In this case the result clause will in general have more literals than $C_1 \cup \{\; p\; \}$ or $C_2 \cup \{\; \neg p\; \}$ alone. Because if the clause $C_1 \cup \{\; p\; \}$ contains $m + 1$ literals and $C_1 \cup \{\; p\; \}$ contains $n + 1$ literals, then the union $C_1 \cup C_2$ can have upto $m + n$ literals in total. In general $m + n$ is bigger than $m + 1$ or $n + 1$. The clause can be smaller than previously if both sets contain the same literals, but it is only granted to be smaller if $m = 0 \lor n = 0$. This is the case for unit clauses. Therefore we will only allow the use of the cut rule if one of the clauses is a unit clause. These cuts will be called unit cuts. To be able to do those unit cuts with a unit clause $\{\; l\; \}$ on all clauses of a set of clauses $K$, we define the function
\\[0.2cm]
\hspace*{1.3cm} $\texttt{unitCut}: 2^{K} \times L \to 2^{K}$
\\[0.2cm]
so that for a set of clauses $K$ and a literal $l$ the function $\texttt{unitCut}(K,\; l)$ will simplify the set of clauses as much as possible by using unit cuts:
\\[0.2cm]
\hspace*{1.3cm} $\texttt{unitCut}(K,\; l) = \{\; C \setminus \{\; \bar{l}\; \}\; |\; C \in K\; \}$
\\[0.2cm]
The result set of clauses will have the same amount of clauses as $K$. Only the clauses $C \in K$ where $\bar{l} \in C$ were affected and got that element removed. \cite{Stroetman2019}

\subsection{Simplification with Subsumption}
\label{sub:sciDavisPutnamSubsumption}
For further simplification of the set of clauses we will use subsumption. The idea is simple and can be shown with the following example:
\\[0.2cm]
\hspace*{1.3cm} $K = \{\; \{\; p,\; \neg q \;\},\; \{\; p\; \}\; \} \cup M$
\\[0.2cm]
It is clear that the clause $\{\; p\; \}$ implies the clause $\{\; p,\; \neg q \;\}$, because if $\{\; p\; \}$ is satisfied, then $\{\; p,\; \neg q \;\}$ will be satisfied too. That's because
\\[0.2cm]
\hspace*{1.3cm} $\models p \to p \lor \neg q$
\\[0.2cm]
is given. With that in mind we can say that we can subsume a clause $C$ with a unit clause $U$ if $U \subseteq C$. That means if we have a set of clauses $K$ with $C \in K \land U \in K$ and $U$ can subsume $C$, we can remove the clause $C$ from $K$ to reduce its size. We define a function
\\[0.2cm]
\hspace*{1.3cm} $\texttt{subsume}: 2^{K} \times L \to 2^{K}$
\\[0.2cm]
that receives a set of clauses $K$ and a literal $l$. It will simplify $K$ by subsuming with the unit clause $\{\; l\; \}$:
\\[0.2cm]
\hspace*{1.3cm} $\texttt{subsume}(K,\; l) = \{\; C \in K\; |\; l \notin C\} \cup \{\{\; l\; \}\}$
\\[0.2cm]
We have to add the unit clause to the set of clauses, because $l \in U$ is the case. Resulting in $U$ being removed from $K$ at first too. \cite{Stroetman2019}

\subsection{Simplification with Case Differentiation}
\label{sub:sciDavisPutnamCaseDiv}
% cSpell:disable
% Die Basis für das Prinzip der Fallunterscheidung bildet folgender Satz.
% cSpell:enable
The following theorem forms the basis for the principle of case differentiation.

\paragraph{Theorem}
% cSpell:disable
% Die Klauselmenge K ist genau dann erfüllbar, wenn die Klausel $K \cup \bigl\{\{p\}\bigr\}$ oder $K \cup \bigl\{\{\neg p\}\bigr\}$ erfüllbar ist.
% cSpell:enable
The set of clauses $K$ can only be satisfied if $K \cup \bigl\{\{p\}\bigr\}$ or $K \cup \bigl\{\{\neg p\}\bigr\}$ can be satisfied.

% cSpell:disable
% Für die Vereinfachung wird zu Beginn also eine Aussagenvariable p ausgewählt, die in der Klauselmenge vorkommt. Danach werden die beiden oben genannten Klauselmengen gebildet und versucht, für eine der beiden eine Lösung zu finden. Ist dieser Versuch erfolgreich, ist das Ergebnis automatisch die Lösung von $K$. Falls keine gefunden wird, ist $K$ unlösbar.
% TODO: Keine Ahnung ob hier Beweis einfügen soll
% cSpell:enable
For simplification, a statement variable $p$ is selected at the beginning, which occurs in the clause set. Then the two clause sets mentioned above are formed and an attempt is made to find a solution for one of them. If this attempt is successful, the result is automatically a solution of $K$. If none is found, $K$ is not satisfiable.

\subsection{The Davis-Putnam Method}
\label{sub:sciDavisPutnamMethod}
% cSpell:disable
% Durch die Wissensgrundlage, die zuvor geschaffen wurde, ist es nun möglich, das Vorgehen des \textit{Davis \& Putman algorithm}us zu skizzieren. Mit der Hilfe der Schnittregel und der Subsumption wird die Klauselmenge $K$ soweit es möglich ist, vereinfacht. Falls bereits nach diesem Schritt $K$ trivial ist, ist das Verfahren beendet. Andernfalls wird eine aussagelogische Variable p ausgewählt, die in K vorkommt. Dann wird rekursiv versucht, die Klauselmenge $K \cup \bigl\{\{p\}\bigr\}$ zu lösen, um eine Lösung für K zu finden. Wenn auch hier keine Lösung gefunden wurde, wird dasselbe mit dem negierten $p$ versucht. Scheitert auch dieser Versuch, ist K unlösbar.
% cSpell:enable
The theoretical foundation that was previously created now makes it possible to sketch the procedure of the \textit{Davis \& Putman algorithm}. With the help of the intersection rule and the subsumption, the clause set $K$ is simplified as far as possible. If already after this step $K$ is trivial, the procedure is finished. Otherwise a statement logical variable p is selected, which occurs in $K$. Then a recursive attempt is made to solve the clause set $K \cup \bigl\{\{p\}\bigr\}$ in order to find a solution for $K$. If no solution was found here either, the same is tried with the negated $p$. If this attempt also fails, $K$ is unsatisfiable.\cite{Zhang2000}, \cite{Stroetman2019}
% TODO: find more sources

\begin{listing}[h!]
  \textbf{function} Satisfiable( clause set $S$ ) \textbf{return} boolean\\
    \hspace*{0.5cm} \textit{/* unit propagation */}\\
    \hspace*{0.5cm} \textbf{repeat}\\
      \hspace*{1.0cm} \textbf{for each} unit clause $L$ in $S$ \textbf{do}\\
        \hspace*{1.5cm} \textit{/* unit subsumption */}\\
        \hspace*{1.5cm} delete from $S$ every clause containing $L$\\
        \hspace*{1.5cm} \textit{/* unit resolution */}\\
        \hspace*{1.5cm} delete $\bar{L}$ from every clause of $S$ in which it occurs\\
      \hspace*{1.0cm} \textbf{od}\\
      \hspace*{1.0cm} \textbf{if} $S$ is empty \textbf{then}\\
        \hspace*{1.5cm} \textbf{return} true\\
      \hspace*{1.0cm} \textbf{else if} a clause becomes $\{\}$ in $S$ \textbf{then}\\
        \hspace*{1.5cm} \textbf{return} false\\
      \hspace*{1.0cm} \textbf{fi}\\
    \hspace*{0.5cm} \textbf{until} no further changes result\\
    \hspace*{0.5cm} \textit{/* splitting */}\\
    \hspace*{0.5cm} choose a literal $L$ occurring in $S$\\
    \hspace*{0.5cm} \textbf{if} Satisfiable ($S \cup \{\{\bar{L}\}\}$) \textbf{then}\\
      \hspace*{1.0cm} \textbf{return} true\\
    \hspace*{0.5cm} \textbf{else if} Satisfiable ($S \cup \{\{L\}\}$) \textbf{then}\\
      \hspace*{1.0cm} \textbf{return} true\\
    \hspace*{0.5cm} \textbf{else}\\
      \hspace*{1.0cm} \textbf{return} false\\
    \hspace*{0.5cm} \textbf{fi}\\
  \textbf{end function}
  \caption{A simple Davis–Putnam algorithm \cite{Zhang2000}}
  \label{code:recursiveDavisPutnam}
\end{listing}

\section{N-Queens Problem}
\label{sec:sciQueens}
% cSpell:disable
% Das n Königinnen Problem ist das verallgemeinerte mathematische Problem bezogen auf ein Schachbrett, dass n \times n Felder besitzt. Ein spezielles Beispiel dazu wäre das 8 Damen Problem, welches auf dem standardisierten Schachbrett abgebildet wird. Allgemein handelt das Problem davon, n Damen auf einem n \times n Schachbrett so zu platzieren, dass keine in ihrem Zug behindert werden würde. Eine Königin in einem normalen Schachspiel kann sowohl diagonal, als auch vertikal und horizontal ihre Züge ziehen. Dieses Zugmuster kann in Figure \label{fig:queens-problem} genauer angeschaut werden. Zusammengefasst heißt dies also, dass immer nur eine Dame auf ihrer vertikalen, horizontalen und diagonalen Linie sein darf, um sich nicht gegenseitig zu behindern. Dabei wird in diesem Problem davon ausgegangen, dass jede Dame jede andere angreifen kann und die Feldfarben ignoriert werden. 
% cSpell:enable
The n queen problem is the generalized mathematical problem related to a chessboard that consists of $n \times n$ squares. A special example would be the 8 queens problem, which is related to the standardized chessboard. In general, the problem is to place n queens on an $n \times n$ chessboard so that none would be obstructed in their turn. A queen in a normal game of chess can move diagonally, vertically and horizontally. This move pattern can be seen in Figure \label{fig:queens-problem}. In summary, this means that there is only one queen allowed on her vertical, horizontal and diagonal line at a time so that they do not interfere with each other. In this problem it is assumed that any queen can attack any other queen and the field colors are ignored. This problem can be solved by several algorithms such as the \textit{Davis \& Putman algorithm}. \cite{Bell2009}, \cite{Watkins2012}, \cite[146\psq]{Nudelman1995}, \cite{Stroetman2019}

\begin{figure}[!ht]
  \centering
  \setlength{\unitlength}{1.0cm}
  \begin{picture}(10,9)
    \thicklines
    \put(1,1){\line(1,0){8}}
    \put(1,1){\line(0,1){8}}
    \put(1,9){\line(1,0){8}}
    \put(9,1){\line(0,1){8}}
    \put(0.9,0.9){\line(1,0){8.2}}
    \put(0.9,9.1){\line(1,0){8.2}}
    \put(0.9,0.9){\line(0,1){8.2}}
    \put(9.1,0.9){\line(0,1){8.2}}
    \thinlines
    \multiput(1,2)(0,1){7}{\line(1,0){8}}
    \multiput(2,1)(1,0){7}{\line(0,1){8}}
    \put(4.15,6.15){{\chess Q}}
    \multiput(5.25,6.5)(1,0){4}{\vector(1,0){0.5}}
    \multiput(3.75,6.5)(-1,0){3}{\vector(-1,0){0.5}}
    \multiput(5.25,7.25)(1,1){2}{\vector(1,1){0.5}}
    \multiput(5.25,5.75)(1,-1){4}{\vector(1,-1){0.5}}
    \multiput(3.75,5.75)(-1,-1){3}{\vector(-1,-1){0.5}}
    \multiput(3.75,7.25)(-1,1){2}{\vector(-1,1){0.5}}
    \multiput(4.5,7.25)(0,1){2}{\vector(0,1){0.5}}
    \multiput(4.5,5.75)(0,-1){5}{\vector(0,-1){0.5}}
  \end{picture}
  \vspace*{-1.0cm}
  \caption{The 8-Queens-Problem with one queen placed \cite{Stroetman2019}} 
  \label{fig:queens-problem}
\end{figure}

\subsection{Define N-Queens Problem for the Davis \& Putnam Algorithm}
Before we can find a solution for the \textit{N-Queens Problem}, we need to define the problem as a propositional logic formula. Since we will use variables that are either true or false, we say that if a variable is true we place a queen and if it is false we are not able to place a queen. As for the naming of these variables we will think about that in the implementation in chapter \ref{ch:implementation}. For now we will say that $Q_{x,y}$ is the queen in row $x$ on column $y$. Now all that is left is the creation of the set of clauses $K$ that describes the \textit{N-Queens Problem}. Therefore we create some functions to help us define the problem as such:

\paragraph{atMostOne}
The function atMostOne will receive a set of literals $l$ and return a set of clauses $K$ so that only one of the literals $l \in L$ can be true. We define it to support following functions.
\\[0.2cm]
\hspace*{1.3cm}$\texttt{atMostOne}:\ L \to K$
\\[0.2cm]
\hspace*{1.3cm}$\texttt{atMostOne}(L) := \{\ \{\ l_1,\ \bar{l_2} \}\ :\ l_1, l_2 \in L\ |\ l_1 \neq l_2\ \}$

\paragraph{atMostOneInRow}
This function receives a number $n$ for the total amount of queens and a number $r \in \{1 .. n\}$ for the row and return a set of clauses $K$ where only one queen can be placed in row $r$. Since only one queen can be placed per row.
\\[0.2cm]
\hspace*{1.3cm} $\texttt{atMostOneInRow}:\ N \times N \to K$
\\[0.2cm]
\hspace*{1.3cm} $\texttt{atMostOneInRow}(r,\ n) := \texttt{atMostOne}(\{\ Q_{r,i} :\ i \in \{1\ ..\ n\}\ \})$

\paragraph{oneInColumn}
This function receives a number $n$ for the total amount of queens and a number $c$ where $c \leq n$ for the column and returns a set of clauses $K$ where at least one queen is placed in the column $c$. Because if only one queen can be placed per row, every column needs at least one queen to have $n$ queens on the chessboard.
\\[0.2cm]
\hspace*{1.3cm} $\texttt{oneInColumn}:\ N \times N \to K$
\\[0.2cm]
\hspace*{1.3cm} $\texttt{oneInColumn}(c,\ n) := \{\ \{\ Q_{i,c} :\ i \in \{1\ ..\ n\}\ \}\ \}$

\paragraph{atMostOneInUpperDiagonal}
This function returns a set of clauses $K$ where only one queen can be placed in the upper diagonal, for a given number $n$ and $k \in \{3\ ..\ 2 * n - 1\}$. The number $k$ helps with calculating which combination of row and column is part of the diagonal.
\\[0.2cm]
\hspace*{1.3cm} $\texttt{atMostOneInUpperDiagonal}:\ N \times N \to K$
\\[0.2cm]
\hspace*{1.3cm} $\texttt{atMostOneInUpperDiagonal}(k,\ n) := \texttt{atMostOne}(\{\ Q_{x,y} :\ x,y \in \{1\ ..\ n\}\ |$
\hspace*{11.5cm} $\ x + y = k\ \})$

\paragraph{atMostOneInLowerDiagonal}
This function returns a set of clauses $K$ where only one queen can be placed in the lower diagonal, for a given number $n$ and $k \in \{-(n - 2)\ ..\ n - 2\}$. As before the number $k$ helps with calculating which combination of row and column is part of the diagonal.
\\[0.2cm]
\hspace*{1.3cm} $\texttt{atMostOneInLowerDiagonal}:\ N \times N \to K$
\\[0.2cm]
\hspace*{1.3cm} $\texttt{atMostOneInLowerDiagonal}(k,\ n) := \texttt{atMostOne}(\{\ Q_{x,y} :\ x,y \in \{1\ ..\ n\}\ |$
\hspace*{11.5cm} $\ x - y = k\ \})$

\paragraph{queensClauses}
With the functions above, we finally can define the \textit{N-Queens Problem} for a given numer $n$ as a propositional logic formula in form of a set of clauses $K$.
\\[0.2cm]
\hspace*{1.3cm} $\texttt{queensClauses}:  N \to K$
\\[0.2cm]
\hspace*{1.3cm} $\texttt{queensClauses}(n) := \bigcup\limits_{i=1}^{n}(\texttt{atMostOneInRow}(i, n)\ \cup\ $
%\\[0.1cm]
%\hspace*{6.5cm}
$\texttt{oneInColumn}(i, n))\ \cup\ $
\\[0.1cm]
\hspace*{6.1cm} $\bigcup\limits_{i=3}^{2 * n - 1}\ \texttt{atMostOneInUpperDiagonal}(i, n)\ \cup\ $
\\[0.1cm]
\hspace*{5.8cm} $\bigcup\limits_{i=-(n - 2)}^{n - 2}\texttt{atMostOneInLowerDiagonal}(i, n)$