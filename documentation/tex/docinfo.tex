% -------------------------------------------------------
% Daten für die Arbeit
% Wenn hier alles korrekt eingetragen wurde, wird das Titelblatt
% automatisch generiert. D.h. die Datei titelblatt.tex muss nicht mehr
% angepasst werden.

\newcommand{\dhbwsprache}{en} % de oder en für Deutsch oder Englisch
                              % Für korrekt sortierte Literatureinträge, noch preambel.tex anpassen

% Titel der Arbeit auf Deutsch
\newcommand{\dhbwtitelde}{Animation des N-Damen Problems in JavaScript}

% Titel der Arbeit auf Englisch
\newcommand{\dhbwtitelen}{Animation of N-Queens Problem in JavaScript}

% Weitere Informationen zur Arbeit
\newcommand{\dhbwort}{Mannheim}    % Ort
\newcommand{\dhbwautorvname}{Koen} % Vorname(n)
\newcommand{\dhbwautornname}{Logman} % Nachname(n)
\newcommand{\dhbwautorrvname}{Jessica} % Vorname(n)
\newcommand{\dhbwautorrnname}{Roth} % Nachname(n)
\newcommand{\dhbwdatum}{\today} % Datum der Abgabe
\newcommand{\dhbwjahr}{2019} % Jahr der Abgabe
\newcommand{\dhbwfirma}{} % Firma bei der die Arbeit durchgeführt wurde
\newcommand{\dhbwbetreuer}{Prof. Dr. Karl Stroetmann, DHBW Mannheim} % Betreuer an der Hochschule
\newcommand{\dhbwzweitkorrektor}{} % Betreuer im Unternehmen oder Zweitkorrektor
\newcommand{\dhbwfakultaet}{I} % I für Informatik
\newcommand{\dhbwstudiengang}{AI} % IB IMB UIB IM MTB

% Zustimmung zur Veröffentlichung
\setboolean{dhbwpublizieren}{true}   % Einer Veröffentlichung wird zugestimmt
\setboolean{dhbwsperrvermerk}{false} % Die Arbeit hat keinen Sperrvermerk

% -------------------------------------------------------
% Abstract

% Kurze (maximal halbseitige) Beschreibung, worum es in der Arbeit geht auf Deutsch
\newcommand{\dhbwabstractde}{
Diese Arbeit befasst sich mit der Erstellung einer Animation vom Davis \& Putnam Algorithmus beim Lösen des N-Damen Problems in JavaScript. Zuerst werden die wissenschaftlichen Grundlagen, die für das Verständnis des Algorithmuses und dem genannten Problem benötigt werden, vermittelt. Dazu kommen noch ein paar technische Grundlagen, die für die Implementierung in JavaScript relevant sind. Nachdem alle Grundlagen bekannt sind, wird zuerst ein Konzept erstellt, welches dann bei der Implementation befolgt wird. Das Ergebnis daraus wird zum Schluss evaluiert und bezüglich möglicher Verbesserungen und anderen Verwendungen betrachtet.
}

% Kurze (maximal halbseitige) Beschreibung, worum es in der Arbeit geht auf Englisch
\newcommand{\dhbwabstracten}{
This work is about the creation of an animation of the Davis \& Putnam algorithm solving the N-Queens Problem in JavaScript. Starting with the scientific basics needed to understand the algorithm and said problem, we continue with some technical basics regarding JavaScript that will affect the implementation. With the foundation set, we continue with a conception and the realization in JavaScript. At the end we evaluate the result, making a conclusion and looking ahead to possible further improvements or applications of said result.
}
