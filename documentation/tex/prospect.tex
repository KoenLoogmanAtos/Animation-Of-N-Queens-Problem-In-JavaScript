% !TeX root = ./documentation.tex

\chapter{Prospect}
\section{Evaluation}
\section{Conclusion}
Wie gerade im vorherigen Abschnitt gezeigt wurde, erfüllt dieses Programm alle zuvor definierten Anforderungen. Damit gilt das Programm als erfolgreich und das Projekt kann nun zufriedenstellend abgeschlossen werden. Zusammenfassend lässt sich sagen, dass es ein sehr spannendes Projekt mit keinen nennenswerten Problemen war. Immer wieder musste im Laufe der Implementierungsphase die Anforderungen erneut betrachtet und mithilfe dieser Entscheidungen getroffen werden. Das Programm sollte zu jeder Zeit leicht verwendbar und verständlich sein. Sowohl Usability als auch User Experience spielten in allen Bereichen des Entwicklungsprozesses eine Rolle. 
\\
Durch die Verwendung von Javascript traten einige Erleichterungen in der Webentwicklung, aber auch einige Schwierigkeiten auf. Diese wurden aber durch hilfreiche Bibliotheken und Technologien wie beispielsweise die Web workers gelöst. 
\section{Outlook} 
%TODO Koen. Du musst sagen, wie man die Berechnung optimieren könnte
Es gibt immer Bereiche eines Programms, die zukünftig noch verbessert werden können. Beispielsweise könnte die Berechnung der Lösung des N Damen Problems bestimmt noch weiter optimiert werden. Dadurch würde die Performanz eventuell verbessert werden, da vor allem bei großen Zahlen für N die Berechnung sehr lange dauert. 
%TODO Koen: durchlesen und eventuell ergänzen
Eine weitere Idee zur zukünftigen Weiterentwicklung wäre die Verwendung dieses Programms für weitere aussagelogische Probleme und nicht nur das N Damen Problem. Durch die Verwendung der web worker wäre dies durch aus denkbar und umsetzbar. SOmit ist der Davis Putman Algorithmus in diesem Programm universell einsetzbar.