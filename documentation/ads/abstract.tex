%!TEX root = ../dokumentation.tex
%\pagestyle{headings}

\chapter*{Abstract}
\subsubsection{Animation of the N-Queens Problem in JavaScript}
This work is about the animation of the Davis-Putnam algorithm solving the N-Queens Problem in JavaScript. Starting with the scientific basics needed to understand the algorithm and said problem, we continue with some technical basics regarding JavaScript that will affect the implementation. With the foundation set, we continue with a conception and the realization in JavaScript. At the end we evaluate the result, making a conclusion and looking ahead to possible further improvements or applications of said result.

\subsubsection{Animation des  N-Damen Problems in JavaScript}
Diese Arbeit befasst sich mit der Animation des Davis-Putnam Algorithmus beim Lösen des N-Damen Problems in JavaScript. Zuerst werden die wissenschaftlichen Grundlagen, die für das Verständnis des Algorithmuses und dem genannten Problem benötigt werden, vermittelt. Dazu kommen noch ein paar technische Grundlagen, die für die Implementierung in JavaScript relevant sind. Nachdem alle Grundlagen bekannt sind, wird zuerst ein Konzept erstellt, welches dann bei der Implementation befolgt wird. Das Ergebnis daraus wird zum Schluss evaluiert und bezüglich möglicher Verbesserungen und anderen Verwendungen betrachtet.